% Options for packages loaded elsewhere
% Options for packages loaded elsewhere
\PassOptionsToPackage{unicode}{hyperref}
\PassOptionsToPackage{hyphens}{url}
\PassOptionsToPackage{dvipsnames,svgnames,x11names}{xcolor}
%
\documentclass[
  11pt,
  letterpaper,
  DIV=11,
  numbers=noendperiod]{scrartcl}
\usepackage{xcolor}
\usepackage[margin = 1in]{geometry}
\usepackage{amsmath,amssymb}
\setcounter{secnumdepth}{-\maxdimen} % remove section numbering
\usepackage{iftex}
\ifPDFTeX
  \usepackage[T1]{fontenc}
  \usepackage[utf8]{inputenc}
  \usepackage{textcomp} % provide euro and other symbols
\else % if luatex or xetex
  \usepackage{unicode-math} % this also loads fontspec
  \defaultfontfeatures{Scale=MatchLowercase}
  \defaultfontfeatures[\rmfamily]{Ligatures=TeX,Scale=1}
\fi
\usepackage{lmodern}
\ifPDFTeX\else
  % xetex/luatex font selection
\fi
% Use upquote if available, for straight quotes in verbatim environments
\IfFileExists{upquote.sty}{\usepackage{upquote}}{}
\IfFileExists{microtype.sty}{% use microtype if available
  \usepackage[]{microtype}
  \UseMicrotypeSet[protrusion]{basicmath} % disable protrusion for tt fonts
}{}
\usepackage{setspace}
\makeatletter
\@ifundefined{KOMAClassName}{% if non-KOMA class
  \IfFileExists{parskip.sty}{%
    \usepackage{parskip}
  }{% else
    \setlength{\parindent}{0pt}
    \setlength{\parskip}{6pt plus 2pt minus 1pt}}
}{% if KOMA class
  \KOMAoptions{parskip=half}}
\makeatother
% Make \paragraph and \subparagraph free-standing
\makeatletter
\ifx\paragraph\undefined\else
  \let\oldparagraph\paragraph
  \renewcommand{\paragraph}{
    \@ifstar
      \xxxParagraphStar
      \xxxParagraphNoStar
  }
  \newcommand{\xxxParagraphStar}[1]{\oldparagraph*{#1}\mbox{}}
  \newcommand{\xxxParagraphNoStar}[1]{\oldparagraph{#1}\mbox{}}
\fi
\ifx\subparagraph\undefined\else
  \let\oldsubparagraph\subparagraph
  \renewcommand{\subparagraph}{
    \@ifstar
      \xxxSubParagraphStar
      \xxxSubParagraphNoStar
  }
  \newcommand{\xxxSubParagraphStar}[1]{\oldsubparagraph*{#1}\mbox{}}
  \newcommand{\xxxSubParagraphNoStar}[1]{\oldsubparagraph{#1}\mbox{}}
\fi
\makeatother


\usepackage{longtable,booktabs,array}
\usepackage{calc} % for calculating minipage widths
% Correct order of tables after \paragraph or \subparagraph
\usepackage{etoolbox}
\makeatletter
\patchcmd\longtable{\par}{\if@noskipsec\mbox{}\fi\par}{}{}
\makeatother
% Allow footnotes in longtable head/foot
\IfFileExists{footnotehyper.sty}{\usepackage{footnotehyper}}{\usepackage{footnote}}
\makesavenoteenv{longtable}
\usepackage{graphicx}
\makeatletter
\newsavebox\pandoc@box
\newcommand*\pandocbounded[1]{% scales image to fit in text height/width
  \sbox\pandoc@box{#1}%
  \Gscale@div\@tempa{\textheight}{\dimexpr\ht\pandoc@box+\dp\pandoc@box\relax}%
  \Gscale@div\@tempb{\linewidth}{\wd\pandoc@box}%
  \ifdim\@tempb\p@<\@tempa\p@\let\@tempa\@tempb\fi% select the smaller of both
  \ifdim\@tempa\p@<\p@\scalebox{\@tempa}{\usebox\pandoc@box}%
  \else\usebox{\pandoc@box}%
  \fi%
}
% Set default figure placement to htbp
\def\fps@figure{htbp}
\makeatother





\setlength{\emergencystretch}{3em} % prevent overfull lines

\providecommand{\tightlist}{%
  \setlength{\itemsep}{0pt}\setlength{\parskip}{0pt}}



 


\KOMAoption{captions}{tableheading}
\usepackage{amsmath}
\usepackage{float}
\makeatletter
\@ifpackageloaded{caption}{}{\usepackage{caption}}
\AtBeginDocument{%
\ifdefined\contentsname
  \renewcommand*\contentsname{Table of contents}
\else
  \newcommand\contentsname{Table of contents}
\fi
\ifdefined\listfigurename
  \renewcommand*\listfigurename{List of Figures}
\else
  \newcommand\listfigurename{List of Figures}
\fi
\ifdefined\listtablename
  \renewcommand*\listtablename{List of Tables}
\else
  \newcommand\listtablename{List of Tables}
\fi
\ifdefined\figurename
  \renewcommand*\figurename{Figure}
\else
  \newcommand\figurename{Figure}
\fi
\ifdefined\tablename
  \renewcommand*\tablename{Table}
\else
  \newcommand\tablename{Table}
\fi
}
\@ifpackageloaded{float}{}{\usepackage{float}}
\floatstyle{ruled}
\@ifundefined{c@chapter}{\newfloat{codelisting}{h}{lop}}{\newfloat{codelisting}{h}{lop}[chapter]}
\floatname{codelisting}{Listing}
\newcommand*\listoflistings{\listof{codelisting}{List of Listings}}
\makeatother
\makeatletter
\makeatother
\makeatletter
\@ifpackageloaded{caption}{}{\usepackage{caption}}
\@ifpackageloaded{subcaption}{}{\usepackage{subcaption}}
\makeatother
\usepackage{bookmark}
\IfFileExists{xurl.sty}{\usepackage{xurl}}{} % add URL line breaks if available
\urlstyle{same}
\hypersetup{
  pdftitle={STATS/CSE 780 - Homework Assignment 1},
  pdfauthor={Toktam Rashidi Shahri (Student Number: XXX)},
  colorlinks=true,
  linkcolor={blue},
  filecolor={Maroon},
  citecolor={Blue},
  urlcolor={Blue},
  pdfcreator={LaTeX via pandoc}}


\title{STATS/CSE 780 - Homework Assignment 1}
\author{Toktam Rashidi Shahri (Student Number: XXX)}
\date{2025-09-21}
\begin{document}
\maketitle


\setstretch{1.5}
\newpage

\subsubsection{Dataset Sourcing}\label{dataset-sourcing}

As indicated in the assignment description, I used the
\href{https://search.open.canada.ca/opendata/}{Open Government Portal}
with filters \emph{Open Data}, \emph{Open Maps}, \emph{Datatset},
\emph{GeoTif}, and \emph{Tif}. The search gave me over 300 results, from
which I chose
\href{https://open.canada.ca/data/en/dataset/ba2645d5-4458-414d-b196-6303ac06c1c9}{Annual
Crop Inventory}. This data is punlished annually in GeoTif format, with
a spatial resolution of 30m pixels, subdivided at provincial boundaries.
My variables of interest are \emph{time}, \emph{crop
density}\footnote{Crop density is not explicitly given in the datasets,
  however, the spatial density of each crop can be calculated using that
  crop's pixel count divided by the total pixel count of all crops in
  the province.}, and \emph{crop type}. Per type, I have chosen to focus
on corn and soybeans.

\subsubsection{Application Description}\label{application-description}

I decided to use the crop data for 10 consecutive years (2015-2024) to
\textbf{study the spatial and temporal change of the two most widely
grown crops in Onrario: \emph{corn} and \emph{soybean}}. I chose the
Provice Ontario because it is the most agricultrally active province in
Canada, and it is also interesting to me as it is where I live.
Moreover, I chose corn and soybean not only becase they are widely
grown, but also because our nutriotion strogly depends on them in terms
of grains, sugars, and oil.

\subsubsection{Data Transformation and
Preprocessing}\label{data-transformation-and-preprocessing}

\begin{enumerate}
\def\labelenumi{\arabic{enumi}.}
\tightlist
\item
  I downloaded the Tif files for Ontario for the 2015 to 2024 period.
  Each file is about 1GB, and contains the crop data for all crop times
  in the provice, as well as data for lands that are not used for
  agriculture, or where clouds made it impossible to determine the crop
  type. I do not need all this data, so I filtered it and made new
  dataframes which I later used for my analysis.
\item
  I used the `rasterio' package to read the files, and converted the
  coordinates to longitude and latitude.
\item
  I am interested in corn and soybean, so I filtered the data for each
  year to oly contain the pixels for these two crops. So, for each year,
  I have a dataframe with 3 colums: longitude, latitude, and crop type
  (corn or soybean). There was not missing data for crops (in the
  original data, there are pixels where it was impossible to determine
  crop type because of clouds, these areas have a different code and are
  already eliminated with the initial filtering).
\item
  I am also interested in the spatial density of these crops for each
  year, so I made a new dataframe that contains year, total crop pixel
  count, corn pixel count, and soybean pixel count. Using this
  dataframe, I will later calculate the spatial density for each crop
  per year. Here, I ignored the lands that were not used for
  agriculture, as well as the lands covered by clouds (which could be
  used for agriculture, but is essentially missing data).
\end{enumerate}

\subsubsection{Single Variable Analysis}\label{single-variable-analysis}

\subparagraph{\texorpdfstring{\textbf{Questions:}}{Questions:}}\label{questions}

\begin{enumerate}
\def\labelenumi{\arabic{enumi}.}
\tightlist
\item
  How has the density of corn changed from 2015 to 2024 in Ontario?
\item
  How has the spatial coverage of corn changed from 2015 to 2024 in
  Ontario?
\end{enumerate}

\subsubsection{Multi Variable Analysis}\label{multi-variable-analysis}

\subparagraph{\texorpdfstring{\textbf{Questions:}}{Questions:}}\label{questions-1}

\begin{enumerate}
\def\labelenumi{\arabic{enumi}.}
\tightlist
\item
  How has the density of corn compare yearly to that of soybean from
  2015 to 2024 in Ontario?
\item
  Is there any spatial relationship between the locations of corn and
  soybean crops in Ontario in 2024?
\end{enumerate}

\subsubsection{Shiny App}\label{shiny-app}

\newpage

\subsection{References}\label{references}

\newpage

\subsection{Supplementary material}\label{supplementary-material}

\begin{enumerate}
\def\labelenumi{\arabic{enumi}.}
\item
  The code that would have interrupted the flow of the body of the
  assignment.
\item
  Reproducible.
\end{enumerate}

We use \texttt{cars} dataset in \texttt{openml} to predict mile per
gallon (mpg) using weight of the car (weight). First, we split the
\texttt{cars} dataset into train/test sets. Then, we use
\texttt{scit-learn} to fit a linear regression model on the train set.
Next, we use compute mean squared error of prediction on the test set.
Finally, we do diagnostics of the fit.

Note: We show only the codes used to get the results in the report. We
can use \texttt{echo\ =\ FLASE} to hide the other codes in the
supplementary material.




\end{document}
